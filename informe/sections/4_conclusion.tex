\section{Conclusiones}
\label{conclusion}

La presente tesina planteó una interrogante acerca del problema de la comprobación de integridad de archivos sobre redes. En primer lugar, se discutieron los inconvenientes inherentes a la centralización actual de dicha prueba. Luego, se brindó un contexto histórico en torno al surgimiento y a la filosofía de la descentralización, dando lugar a un nuevo escenario para presentar una nueva alternativa de solución a este problema. Tras ello, se desarrolló el marco teórico de blockchain, mostrando sus conceptos originarios para luego explicar las dos implementaciones más grandes y populares: Bitcoin y Ethereum. En la primera se explicaron las ideas fundamentales que son comunes a prácticamente toda blockchain, en particular, las transacciones, los bloques, el funcionamiento de la red y la política de consenso basada en prueba de trabajo. En la segunda se extendieron estas componentes para incluir la posibilidad de realizar cómputo descentralizado por medio de contratos inteligentes, con los cuales, se podrían desarrollar aplicaciones genéricas sobre la blockchain, más allá del uso estrictamente monetario para el cual fue originalmente diseñado Bitcoin. Finalmente, se presentó una arquitectura piloto que describe la solución alterna al problema de la verificación de archivos haciendo uso de los mencionados contratos inteligentes.

Se ha visto como ventajas que la nueva solución provee una fuente de información confiable basada en la inmutabilidad de los datos una vez que son confirmados dentro de la blockchain y accesible desde cualquier nodo conformante de dicha red. A diferencia de las soluciones centralizadas tradicionales, las vías de ataque son drásticamente reducidas dado que no alcanza con alterar maliciosamente o denegar los datos registrados por uno de los nodos, puesto que hay redundancia en todos ellos. Por otra parte, dadas las demoras necesarias para confirmar transacciones por los mecanismos de consenso, en general toda solución basada en blockchain será más lenta. Además, o por lo menos en la implementación realizada en este trabajo, es necesario contar con una billetera y dinero digital para poder realizar las operaciones. En el apartado ``Trabajos futuros'' (\ref{future_works}) se explicarán algunos mecanismos para mitigar estos inconvenientes.

Respecto a las expectativas antes y después de esta investigación, se pueden mencionar:

\begin{enumerate}
  \item Antes de comenzar a recabar y analizar información sobre el consenso en Bitcoin, los tesinistas creían que dicho algoritmo era totalmente tolerante a fallas bizantinas. A medida que se avanzó con esto, se comenzó a ver que la premisa anterior no era del todo verdadera y que dicha tolerancia es parcial. Es decir, a diferencia de otros algoritmos, como por ejemplo PBFT, que demostraron ser completamente tolerantes a fallas bizantinas, los mecanismos de consenso usados en Bitcoin y Ethereum (al día de hoy) -basados en prueba de trabajo-, son probabilísticamente tolerantes a dichas fallas, pero en la práctica han mostrado comportarse muy bien y, además, son anónimos por diseño (en tanto que PBFT no lo es) lo cual es fundamental en implementaciones de blockchains públicas.
  \item Eliminar completamente la centralización es un trabajo mayor del pensado. Por ejemplo, si se analiza el prototipo realizado para esta tesina, la aplicación cliente -frontend- y el servidor de archivos actualmente residen en un servidor centralizado y podrían ser vulnerable a ataques. Asimismo, la conexión del frontend a la blockchain es punto a punto y dicho canal también podría ser atacado. De todas formas, estos problemas son abordables, y en ``Trabajos Futuros'' (\ref{future_works}) se discutirán algunas soluciones.
  \item El desarrollo del prototipo tuvo sus dificultades puesto que las librerías a usar aún están en proceso de desarrollo constante, con lo cual, la interfaz de muchas de ellas es algo inestable. De hecho, se ha contribuido a mejorar algunas de estas herramientas open source.\footnote{\url{https://github.com/ConsenSys/abi-decoder/pull/29}}.
\end{enumerate}

Otra cuestión a destacar es que más allá de las implementaciones vistas de blockchain en este trabajo, existen centenares más, donde cada una busca solucionar un problema específico y concreto. De hecho, existe una extensa clasificación de acuerdo al mecanismo de consenso usado -priorizando uso de energía, tolerancia a ataques, velocidad en cuanto al tiempo promedio en confirmar una transacción y escalabilidad en la red-, a la visibilidad de los datos (si es pública o privada), entre otros. Cada negocio buscará su tipo ideal de blockchain acorde a su dominio.

Asimismo, es interesante mencionar la gran cantidad de líneas de investigación y desarrollo que se pueden derivar del estudio de blockchain. Por ejemplo, se puede innovar en temáticas tales como teoría de lenguajes (para describir contratos inteligentes), máquinas virtuales, probabilidad y estadística (estudio de la robustez de los mecanismos de consenso y resistencia a ataques), criptografía, sistemas distribuidos (protocolos de consenso, en particular) y hasta en aspectos económicos y de mercado que escapan de la computación en sí.

Finalmente, vale destacar que el tema desarrollado en la presente tesina ha sido presentado y aceptado en la WICC San Juan 2019, enmarcado dentro de la línea de investigación de ``Seguridad Informática''.
