\section{Trabajos futuros}
\label{future_works}

A lo largo del desarrollo de la tesina se formularon distintas interrogantes en cuanto a la estructura arquitectónica de la solución propuesta. Una de ellas particularmente referida a los conocimientos básicos que debe tener el usuario final, como por ejemplo, la noción de conceptos como el gas consumido por las transacciones que se paga en \textit{ether} y el uso de extensiones para la administración de billeteras (Metamask). Una solución para este tipo de problemas es el uso de estaciones de gas. Las estaciones de gas están conformadas por una serie de componentes que realizan el pago y el envío a la red de las transacciones, de una forma transparente para el usuario final. Las componentes y el rol que ocupan cada una de ellas es:

\begin{enumerate}
  \item El usuario final que posee una cuenta con una clave privada que es utilizada para crear las transacciones y firmarlas (la cuenta no posee \textit{ether}).
  \item Distribuidor de transacciones, es la cuenta encargada de encapsular las transacciones de los usuarios finales y enviarlas a un contrato inteligente para luego ser ejecutada. Esta cuenta debe poseer balance en \textit{ether}.
  \item Un contrato inteligente denominado ejecutor, que es el encargado de ejecutar la transacción del usuario, previamente validando su contenido y calculando la cantidad de gas necesaria para ser ejecutada.
  \item Estación de gas, es un contrato inteligente que posee balance en \textit{ether} o algún otro tipo de criptomoneda que son utilizadas para pagarle al distribuidor de transacciones por el envío de la transacción al contrato ejecutor y su distribución en la red.
\end{enumerate}

Entonces la secuencia de pasos sería la siguiente: el usuario final crea y firma la transacción (utilizando su clave privada) y la envía a la cuenta del distribuidor. Este último posee \textit{ether}, por lo tanto, puede ejecutar la función sobre el contrato ejecutor que le permite enviar la transacción del usuario final. Una vez en el contrato ejecutor, la transacción es verificada y enviada a la red, y a su vez éste calcula el gas que utilizó el distribuidor para realizar todo el proceso y se le reembolsa a través del contrato estación de gas.
De esta manera el usuario no necesita instalar ninguna extensión o aplicación extra para el uso de billeteras, ya que la clave privada se podría crear a la hora del registro del usuario y otra ventaja es que no necesita obtener \textit{ether} para la ejecución de las transacciones.

Otra de las problemáticas planteadas fue el poder eliminar completamente la centralización desde el punto de vista de la instalación y despliegue de la aplicación y de la comunicación con la blockchain. En cuanto a lo primero, se destacó que la existencia de una única instancia tanto de aplicación cliente como de servidor de archivos podría ser un punto débil desde la perspectiva de la seguridad y la disponibilidad del servicio. El mismo problema se plantea con la descarga de la aplicación cliente, puesto que se haría desde un repositorio centralizado. Una forma de atacar estos problemas es usando IPFS\cite{Ipfs2019}, el cual está pensado como sistema de archivos distribuido par-a-par. Si las distintas versiones oficiales de la aplicación cliente se alojasen allí, entonces la descarga de las mismas serían mucho más seguras por estar justamente delegando la confianza sobre la red y, por el contrario, que no recaiga en un único punto de acceso. En cuanto a lo segundo, la comunicación a los nodos, lo ideal es tener un conjunto de nodos totalmente sincronizados y estables, de forma tal que las peticiones o envío de información a la blockchain sean alternados y no dejar de prestar servicio si uno de los nodos se cae, otorgando alta disponibilidad al sistema. Para la solución propuesta en la tesina se utilizó un solo proveedor de acceso, Infura, que brinda una URL de conexión para enviar y recibir los mensajes RPC-JSON. Éste es un servicio gratuito y no es dedicado, por lo que el flujo de entrada y de salida es un tanto congestionado. Mantener varios nodos con la blockchain sincronizados es muy costoso: al momento de escribir la tesina la base de datos de Ethereum pesa alrededor de 224.15 GB. Existen alternativas como QuikNode\cite{Quiknode2019} o Infura en su versión paga, que proveen el acceso a un nodo totalmente sincronizado y dedicado, lo cual es una buena alternativa a la creación de uno propio.

Otro caso de uso que se podría implementar para un trabajo futuro, es la creación de una aplicación descentralizada que certifique y realice el seguimiento de los títulos universitarios dentro de una blockchain pública, o bien, dentro de una privada. Las blockchains privadas son altamente performantes en cuanto a velocidad y costos, donde el precio del gas utilizado para ejecutar las transacciones es muy bajo o incluso, en ciertos casos, nulo.
Las blockchains privadas poseen una forma de gobernanza distinta a las públicas ya que las decisiones de cambios o modificaciones de protocolo se realizan mediante el voto de las partes involucradas. La información que reside dentro de este tipo de blockchains es de acceso público, lo que es privado es el derecho a formar parte de la red mediante la inclusión de un nuevo nodo. Un ejemplo de este tipos de redes es el de la Blockchain Federada Argentina\cite{Bfa2019}, la cual propone una red en la cual distintas instituciones como  gobiernos provinciales, administración pública y instituciones académicas, entre otros, puedan interactuar entre sí e intercambiar información de una manera sencilla y eficiente. Volviendo al caso de uso propuesto, sería interesante implementar dicha aplicación en una blockchain como la anteriormente descrita, ya que reduciría el costo administrativo y los procedimientos burocráticos que requiere la certificación y verificación de los títulos universitarios.

En el caso que dicha implementación crezca, posteriormente se podría solucionar la interacción con otras blockchains similares -por ejemplo, certificaciones a nivel mundial o en distintas regiones del pais- mediante sidechains\cite{Sidechain2019}, el cual es un mecanismo relativamente nuevo para ``traducir'' o establecer alguna clase de equivalencia desde una red hacia otra. Para el caso previamente planteado, esto podría servir para la legalización de títulos extranjeros: una certificación en la blockchain local podría estar vinculada de alguna manera a otra certificación proveniente de una blockchain extranjera y, a través de este enlace, establecer la legalización del título.
