% Comandos personalizados

% {\fechaPresentacion} :: para escribir la fecha de presentación del trabajo
\newcommand{\fechaPresentacion}{\today}
% {\unlp} :: para escribir "Universidad Nacional de La Plata"
\newcommand{\unlp}{Universidad Nacional de La Plata}
% {\facultad} :: para escribir "Facultad de Informática"
\newcommand{\facultad}{Facultad de Informática}
% {\tituloTrabajo} :: Para escribir el título de la tesina
\newcommand{\tituloTrabajo}{Integridad de archivos y su distribución en una red blockchain}
% \tituloTrabajoDosLineas :: Para escribir el título de la tesina en dos líneas (carátula)
\newcommand{\tituloTrabajoDosLineas}{Integridad de archivos\\* y su distribución en una red blockchain}
% {\sebaponti} :: para escribir "Sebastian Ponti"
\newcommand{\sebaponti}{Sebastian Ponti}
% {\pontiseba} :: para escribir "Ponti, Sebastian"
\newcommand{\pontiseba}{Ponti, Sebastian}
% {\lndl} :: para escribir "Lautaro Nahuel De León"
\newcommand{\lndl}{Lautaro Nahuel De León}
% {\dlln} :: para escribir "De León, Lautaro Nahuel"
\newcommand{\dlln}{De León, Lautaro Nahuel}

% {\caratula} :: para generar la carátula de la tesina
\newcommand{\caratula}{
  \begin{center}
    \includegraphics{images/unlp.png}\\
    \huge{\unlp}\\
    \vspace{5mm}
    \huge{\facultad}\\
    \vspace{5mm}
    \large{Tesina de la Licenciatura}\\
    \vspace{15mm}
    \huge{\tituloTrabajoDosLineas}\\
    \vspace{10mm}
    \large{\textbf{\pontiseba} \\
    \textbf{\dlln}}\\
    \vspace{20mm}
    \large{Directora: Dra. Patricia Bazán}\\
    \vspace{20mm}
    \normalsize{\fechaPresentacion}\\
  \end{center}
}

% {\checkmark} :: para imprimir un check (tilde)
\def\checkmark{\tikz\fill[scale=0.4](0,.35) -- (.25,0) -- (1,.7) -- (.25,.15) -- cycle;}
